\documentclass[12pt]{scrartcl}

\usepackage[english]{babel}
\usepackage[sorting=none,backend=biber]{biblatex}
\usepackage{amsmath}
\usepackage{graphicx}
\usepackage[noend]{algpseudocode}
\usepackage{algorithm}
\usepackage{hyperref}
\usepackage[backend=cairo, outputdir=diagrams]{diagrams-latex}

\newtheorem{dfn}{Definition}

\title{Document}
\author{Kai Zhang}
\date{}

\begin{document}

\maketitle

\tableofcontents
\newpage

1-based coordinate system A coordinate system where the first base of a sequence
is one. In this system, a region is specified by a closed interval. For example, the
region between the 3rd and the 7th bases inclusive is [3, 7]. The SAM, VCF, GFF and Wiggle
formats are using the 1-based coordinate system.

0-based coordinate system A coordinate system where the first base of a sequence
is zero. In this coordinate system, a region is specified by a half-closed-half-open interval.
For example, the region between the 3rd and the 7th bases inclusive is [2, 7).
The BAM, BCFv2, BED, and PSL formats are using the 0-based coordinate system.

\printbibliography

\end{document}
